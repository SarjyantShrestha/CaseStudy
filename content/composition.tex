\subsection{Composition of Bricks: Ingredients and Functions}
Bricks, essential for construction, are rectangular units used in walls, pavements, and masonry when stone is scarce. Brick chips also serve as coarse aggregate in concrete mixtures.

\subsubsection{Ingredients and Percentages}
The primary ingredients and their proportions in bricks are:
\begin{itemize}
  \item Silica (SiO2) - 55\%
  \item Alumina (Al2O3) - 30\%
  \item Iron Oxide (Fe2O3) - 8\%
  \item Magnesia (MgO) - 5\%
  \item Lime (CaO) - 1\%
  \item Organic Matter - 1\%
\end{itemize}

\subsubsection{Chief Ingredients and Functions}
\begin{itemize}
  \item Silica and Alumina: Provide plasticity and prevent cracking.
  \item Alumina: Acts as a cementing material for molding.
  \item Silica: Combats cracking, shrinking, and warping.
  \item Lime: Aids fusion, strengthening, and durability.
  \item Iron Oxide: Enhances strength and imperviousness.
  \item Magnesia: Lessens shrinkage and offers color variation.
\end{itemize}

\subsubsection{Harmful Ingredients}
\begin{itemize}
  \item Excess Lime: Melts and disfigures bricks.
  \item Alkalis (e.g., Sodium, Potassium): Cause fusion, dampness, and efflorescence.
  \item Pebbles, Stones, Gravels: Result in weak and irregular bricks.
  \item Iron Pyrites (FeS): Causes crystallization and discoloration.
  \item Organic Matter: Renders bricks porous and weakens production.
\end{itemize}